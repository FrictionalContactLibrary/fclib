\documentclass[a4paper,10pt]{article}
\input{packages.tex}
\usepackage{a4wide}
\newcommand{\fcadre}[1]
        {\begin{center}\fbox{$ \displaystyle {#1} $} \end{center}}
\newcommand{\Fcadre}[1]
        {\fbox{$ \displaystyle {#1} $} }        
\newcommand{\w}[1]{\omega_{#1}}
\newcommand{\brw}[1]{\Breve{\omega}_{#1}}
\newcommand{\brgam}[2]{\Breve{\gamma}_{x_{#1}x_{#2}}}
\newcommand{\bru}[1]{\Breve{u}_{#1}}
\newcommand{\bPI}{\boldsymbol\Pi}
%Operateur de derivee partielle classique
\newcommand{\DP}[2]{\displaystyle \frac{\partial {#1}}{\partial {#2}}}
\newcommand{\DPP}[2]{\displaystyle\frac{\partial^2 {#1}}{\partial {#2}^2}}
%Operateur de derivee partielle d'une parenthese separee
\newcommand{\DS}[2]{\displaystyle\frac{\partial}{\partial {#2}}\left(#1\right)}
\newcommand{\DSS}[2]{\displaystyle\frac{\partial^2}{\partial {#2}^2}\left(#1\right)}
%Operateur de derivee  classique
\newcommand{\DT}[2]{\displaystyle \frac{d\, {#1}}{d\,{#2}}}
\newcommand{\DTT}[2]{\displaystyle\frac{d^2\, {#1}}{d\, {#2}^2}}



\newcommand{\intinf}[1]{\int_{-\infty}^{+\infty} {#1}\,dt}
\newcommand{\INTA}[1]{\int^{A}_{-A}{#1}\,dx}
\newcommand{\INTL}[1]{\int^{L}_{0}{#1}\,dx_{1}}
\newcommand{\DBINT}[1]{\int \int_{\Omega}{#1}\,dx\,dt}
\newcommand{\INTS}[1]{\int_{S} {#1} \,ds}
\def\dx{\,dx}
\def\dt{\,dt}
\def\ds{\,ds}
\def\dnu{\,d\nu}
\def\dtheta{\,d\theta}


\newcommand{\norme}[1]{\|#1\|}
\newcommand{\normeinf}[1]{\|#1\|_{\infty}}
\newcommand{\normeE}[1]{\|#1\|_{\mathcal E}}
\newcommand{\Lnorme}[1]{\norme{#1}_{{Cal L}_2}} 
\newcommand{\scal}[2]{\left\langle{#1},{#2}\right\rangle} 





\newcommand{\etoi}[1]{\stackrel{\star}{#1}}
\newcommand{\contract}{{\,\Bar{\Bar\otimes}\,}}
%\newcommand{\contract}{{\,\overline{\overline\otimes}\,}}
\newcommand{\scontract}{\,{\Bar\otimes}\,}
\newcommand{\ouvert}[1]{\stackrel{\circ}{#1}}
% displaystyle et fraction displayed

%% Symbole de fraction
\newcommand{\Frac}[2]{{\displaystyle \frac{\displaystyle #1}{\displaystyle #2}}}
\newcommand{\Prac}[2]{\displaystyle \genfrac{(}{)}{}{}{\displaystyle #1}{\displaystyle #2}}
\newcommand{\Crac}[2]{\displaystyle \genfrac{[}{]}{}{}{\displaystyle #1}{\displaystyle #2}}
%\newcommand{\binom}[2]{\genfrac{(}{)}{0pt}{}{#1}{#2}}
%\def\bigint{{\displaystyle\int}} 
\def\Int{\displaystyle\int}
%\newcommand{\Int}{\displaystyle\int}




% Symboles mathemeatiques en gras
\newcommand{\baa}{\boldsymbol a }
\newcommand{\baA}{\boldsymbol A }
\newcommand{\bbb}{\boldsymbol b }
\newcommand{\bbB}{\boldsymbol B }
\newcommand{\bcC}{\boldsymbol C }
\newcommand{\bcc}{\boldsymbol c }
\newcommand{\bdD}{\boldsymbol D }
\newcommand{\bdd}{\boldsymbol d }
\newcommand{\bee}{\boldsymbol e }
\newcommand{\beE}{\boldsymbol E }
\newcommand{\bff}{\boldsymbol f }
\newcommand{\bfF}{\boldsymbol F }
\newcommand{\bgg}{\boldsymbol g }
\newcommand{\bgG}{\boldsymbol G }
\newcommand{\bhh}{\boldsymbol h }
\newcommand{\bhH}{\boldsymbol H }
\newcommand{\bii}{\boldsymbol i }
\newcommand{\biI}{\boldsymbol I }
\newcommand{\bll}{\boldsymbol l }
\newcommand{\blL}{\boldsymbol L }
\newcommand{\bmm}{\boldsymbol m}
\newcommand{\bmM}{\boldsymbol M}
\newcommand{\bnn}{\boldsymbol n }
\newcommand{\bnN}{\boldsymbol N}
\newcommand{\bkK}{\boldsymbol K }

\newcommand{\bpp}{\boldsymbol p}
\newcommand{\bpP}{\boldsymbol P}
\newcommand{\bqq}{\boldsymbol q}
\newcommand{\bddq}{\Ddot{\boldsymbol q}}
\newcommand{\bdq}{\Dot{\boldsymbol q}}
\newcommand{\bqQ}{\boldsymbol Q}
\newcommand{\brR}{\boldsymbol R }
\newcommand{\brr}{\boldsymbol r }
\newcommand{\bss}{\boldsymbol s }
\newcommand{\bsS}{\boldsymbol S }
\newcommand{\btt}{\boldsymbol t }
\newcommand{\btT}{\boldsymbol T }
\newcommand{\buu}{\boldsymbol u}
\newcommand{\buU}{\boldsymbol U}
\newcommand{\bvV}{\boldsymbol V}
\newcommand{\bvv}{\boldsymbol v}
\newcommand{\bhvv}{\Hat{\boldsymbol v}}
\newcommand{\bwW}{\boldsymbol W}
\newcommand{\bww}{\boldsymbol w}
\newcommand{\bxX}{\boldsymbol X }
\newcommand{\bxx}{\boldsymbol x }
\newcommand{\byY}{\boldsymbol Y }
\newcommand{\byy}{\boldsymbol y }
\newcommand{\bzZ}{\boldsymbol Z }
\newcommand{\bzz}{\boldsymbol z }

\newcommand{\bdel}{\boldsymbol \delta }
\newcommand{\brho}{\boldsymbol \rho }
\newcommand{\bgam}{\boldsymbol \gamma }
\newcommand{\bsig}{\boldsymbol \sigma }
\newcommand{\beps}{\boldsymbol \epsilon }
\newcommand{\btau}{\boldsymbol \tau }
\newcommand{\btheta}{\boldsymbol \theta }
\newcommand{\balpha}{\boldsymbol \alpha }
\newcommand{\bbeta}{\boldsymbol \beta }
\newcommand{\bomega}{\boldsymbol \omega}
\newcommand{\bOmega}{\boldsymbol \Omega}
\newcommand{\bups}{\boldsymbol \Upsilon}

\newcommand{\mC}{\mathcal C}
\newcommand{\mR}{\mathcal R}
\newcommand{\HRule}{\rule{\linewidth}{1mm}}
\newcommand{\demi}{{\displaystyle\frac{1}{2}}}
\newcommand{\ddemi}{\frac{1}{2}}
\newcommand{\tiers}{\displaystyle\frac{1}{3}}
\newcommand{\quart}{\displaystyle\frac{1}{4}}
\newcommand{\dixi}{\displaystyle\frac{1}{10}}

\newcommand{\bsy}[1]{\boldsymbol{#1}}


% Fonction math�matiques

\newcommand{\divx}{\text{div}_{\boldsymbol x}}
\newcommand{\ddivx}{\text{divdiv}_{\boldsymbol x}}
\newcommand{\divX}{\text{div}_{\boldsymbol X}}
\newcommand{\gradx}{\text{grad}_{\boldsymbol x}}
\newcommand{\sign}{\text{sign}}
\newcommand{\divy}{\text{div}_{\boldsymbol y}}
\newcommand{\transposee}[1]{{\vphantom{#1}}^{\text{\tiny{\textsf T}}}{#1}}
%\newcommand{\transposee}[1]{{\vphantom{#1}}^{\mathit{t}}{#1}}
\newcommand{\invtransposee}[1]{{\vphantom{#1}}^{\text{T}}{#1}}
%\newcommand{\invtransposee}[1]{{\vphantom{#1}}^{\mathit{-t}}{#1}}
\newcommand{\argmin}{\mathop{\mathrm{argmin}}}
\newcommand{\argminn}{\mathop{\mathrm{argmin}}\nolimits}
% macro pour les symbols d'ensemble
%\nbOne
\def\nbOne{{\mathchoice{\rm 1\mskip-4mu l}{\rm 1\mskip-4mu l} {\rm 1 \mskip-4.5mu l}{\rm 1\mskip-5mu l}}}
%
%%  Les ensembles de nombres  C. Fiorio (fiorio�at�math.tu-berlin.de) 
%
\def\nbR{\ensuremath{\mathrm{I\!R}}} % IR
\def\nbN{\ensuremath{\mathrm{I\!N}}} % IN
\def\nbF{\ensuremath{\mathrm{I\!F}}} % IF
\def\nbH{\ensuremath{\mathrm{I\!H}}} % IH
\def\nbK{\ensuremath{\mathrm{I\!K}}} % IK
\def\nbL{\ensuremath{\mathrm{I\!L}}} % IL
\def\nbM{\ensuremath{\mathrm{I\!M}}} % IM
\def\nbP{\ensuremath{\mathrm{I\!P}}} % IP
%
% \nbOne : 1I : symbol one
\def\nbOne{{\mathchoice {\rm 1\mskip-4mu l} {\rm 1\mskip-4mu l}
{\rm 1\mskip-4.5mu l} {\rm 1\mskip-5mu l}}}
%
% \nbC   :  Nombres Complexes
\def\nbC{{\mathchoice {\setbox0=\hbox{$\displaystyle\rm C$}%
\hbox{\hbox to0pt{\kern0.4\wd0\vrule height0.9\ht0\hss}\box0}}
{\setbox0=\hbox{$\textstyle\rm C$}\hbox{\hbox
to0pt{\kern0.4\wd0\vrule height0.9\ht0\hss}\box0}}
{\setbox0=\hbox{$\scriptstyle\rm C$}\hbox{\hbox
to0pt{\kern0.4\wd0\vrule height0.9\ht0\hss}\box0}}
{\setbox0=\hbox{$\scriptscriptstyle\rm C$}\hbox{\hbox
to0pt{\kern0.4\wd0\vrule height0.9\ht0\hss}\box0}}}}
%
% \nbQ   : Nombres Rationnels Q
\def\nbQ{{\mathchoice {\setbox0=\hbox{$\displaystyle\rm
Q$}\hbox{\raise
0.15\ht0\hbox to0pt{\kern0.4\wd0\vrule height0.8\ht0\hss}\box0}}
{\setbox0=\hbox{$\textstyle\rm Q$}\hbox{\raise
0.15\ht0\hbox to0pt{\kern0.4\wd0\vrule height0.8\ht0\hss}\box0}}
{\setbox0=\hbox{$\scriptstyle\rm Q$}\hbox{\raise
0.15\ht0\hbox to0pt{\kern0.4\wd0\vrule height0.7\ht0\hss}\box0}}
{\setbox0=\hbox{$\scriptscriptstyle\rm Q$}\hbox{\raise
0.15\ht0\hbox to0pt{\kern0.4\wd0\vrule height0.7\ht0\hss}\box0}}}}
%
% \nbT   : T
\def\nbT{{\mathchoice {\setbox0=\hbox{$\displaystyle\rm
T$}\hbox{\hbox to0pt{\kern0.3\wd0\vrule height0.9\ht0\hss}\box0}}
{\setbox0=\hbox{$\textstyle\rm T$}\hbox{\hbox
to0pt{\kern0.3\wd0\vrule height0.9\ht0\hss}\box0}}
{\setbox0=\hbox{$\scriptstyle\rm T$}\hbox{\hbox
to0pt{\kern0.3\wd0\vrule height0.9\ht0\hss}\box0}}
{\setbox0=\hbox{$\scriptscriptstyle\rm T$}\hbox{\hbox
to0pt{\kern0.3\wd0\vrule height0.9\ht0\hss}\box0}}}}
%
% \nbS   : S
\def\nbS{{\mathchoice
{\setbox0=\hbox{$\displaystyle     \rm S$}\hbox{\raise0.5\ht0%
\hbox to0pt{\kern0.35\wd0\vrule height0.45\ht0\hss}\hbox
to0pt{\kern0.55\wd0\vrule height0.5\ht0\hss}\box0}}
{\setbox0=\hbox{$\textstyle        \rm S$}\hbox{\raise0.5\ht0%
\hbox to0pt{\kern0.35\wd0\vrule height0.45\ht0\hss}\hbox
to0pt{\kern0.55\wd0\vrule height0.5\ht0\hss}\box0}}
{\setbox0=\hbox{$\scriptstyle      \rm S$}\hbox{\raise0.5\ht0%
\hboxto0pt{\kern0.35\wd0\vrule height0.45\ht0\hss}\raise0.05\ht0%
\hbox to0pt{\kern0.5\wd0\vrule height0.45\ht0\hss}\box0}}
{\setbox0=\hbox{$\scriptscriptstyle\rm S$}\hbox{\raise0.5\ht0%
\hboxto0pt{\kern0.4\wd0\vrule height0.45\ht0\hss}\raise0.05\ht0%
\hbox to0pt{\kern0.55\wd0\vrule height0.45\ht0\hss}\box0}}}}
%
% \nbZ   : Entiers Relatifs Z
\def\nbZ{{\mathchoice {\hbox{$\sf\textstyle Z\kern-0.4em Z$}}
{\hbox{$\sf\textstyle Z\kern-0.4em Z$}}
{\hbox{$\sf\scriptstyle Z\kern-0.3em Z$}}
{\hbox{$\sf\scriptscriptstyle Z\kern-0.2em Z$}}}}
%%%% fin macro %%%%



\newcommand{\putidx}[1]{\index{#1}\textit{#1}}
% macro pour r�f�rencer les �quations

\newcommand{\refeq}[1]{(\ref{#1})}
\newcommand{\reffig}[1]{({\it cf} figure : \ref{#1})}
\newcommand{\refann}[1]{({\it cf} Annexe : \ref{#1})}


%\definecolor{darkgray}{gray}{.25}
%\definecolor{gray}{gray}{.5}
%\definecolor{lightgray}{gray}{.75}
%\definecolor{gradbegin}{rgb}{0,1,1}
%\definecolor{gradend}{rgb}{0,.1,.95}
%\newcommand{\newtexte}[1]{\textcolor{darkgray} {#1}}
\newcommand{\newtexte}[1]{{#1}}% macro pour les varibales favorites
% normal tangent
\def\n{{\hbox{\tiny{N}}}}
\def\t{{\hbox{\tiny{T}}}}
\def\nt{\hbox{\tiny{NT}}}
\def\nsf{\hbox{\tiny{\textsf N}}}
\def\tsf{\hbox{\tiny{\textsf T}}}
\def\sigman{\sigma_{\n}}
\def\sigmat{\sigma_{\t}}
\def\sigmant{\sigma_{\nt}}
\def\epsn{\epsilon_{\n}}
\def\epst{\epsilon_{\t}}
\def\epsnt{\epsilon_{\nt}}
\def\eps{\epsilon}
\def\veps{\varepsilon}
\def\sig{\sigma}
\def\Rn{R_{\n}}
\def\Rt{R_{\t}}
\def\cn{c_{\n}}
\def\Cn{C_{\n}}
\def\ct{c_{\t}}
\def\Ct{C_{\t}}
\def\un{u_{\n}}
\def\ut{\buu_{\t}}
\def\uut{u_{\t}}
\def\unc{u_{\n}^c}
\def\utc{\buu_{\t}^c}
\def\vn{v_{\n}}
\def\vt{v_{\t}}
\def\rr{\hbox{\tiny{\textsf R}}}
\def\irr{\hbox{\tiny{\textsf{IR}}}}
\def\rn{r_{\n}}
\def\rt{\brr_{\t}}
\def\rnc{r_{\n}^c}
\def\rtc{\brr_{\t}^c}
\def\trn{\Tilde{r}_{\n}}
\def\trt{\Tilde{\brr}_{\t}}
\def\tr{\Tilde{\brr}}
\def\tv{\Tilde{\bvv}}
\def\vn{v_{\n}}
\def\vt{\bvv_{\t}}
\def\adh{\mathsf{adh}}
\def\adj{\hbox{\tiny{\textsf{adj}}}}
\def\adjc{\hbox{\tiny{\textsf{adjC}}}}
\def\adja{\hbox{\tiny{\textsf{adjA}}}}
\def\cc{\hbox{\tiny{\textsf C}}}
\def\ca{\hbox{\tiny{\textsf A}}}
%%    Unit�e
\def\mm{\,\mathsf{mm}}
\def\cm{\,\mathsf{cm}}
\def\m{\,\mathsf{m}}
\def\ms{\,\mathsf{m.s^{-1}}}
\def\mms{\,\mathsf{mm.s^{-1}}}
\def\Mpa{\,\mathsf{MPa}}
\def\Gpa{\,\mathsf{GPa}}
\def\Kg{\,\mathsf{Kg}}
\def\Hz{\,\mathsf{Hz}}
\def\kHz{\,\mathsf{kHz}}
\def\N{\,\mathsf{N}}
\def\kN{\,\mathsf{kN}}
\def\Nmmm{\,\mathsf{N.m^{-3}}}
\def\ds{d_{\hbox{\tiny{S}}}}
% domaines et frontieres
\def\om{\Omega}
\def\oma{\Omega^{\alpha}}
\def\omu{\Omega^1\cup \Omega^2}
\def\gc{\Gamma_c}
\def\omt{\omu \cup \gc}
% derivee partielle et gradient et divergence
\def\p{\partial}
\def\grad{\nabla}
\def\div{\mathop{\rm div}\nolimits}
%

%\DeclareTextSymbol{\deg}{T1}{6}
%\def\degre{\mathdegree}
%\newcommand{\degre}{\mathdegree}

\def\etc{\textit{etc}\ldots}
\newcommand{\mdegre}{\hbox{\text{\degre}}}

%\def\nscd{\textsf{\bfseries NSCD}}
%\def\nscd{\textsf{NSCD}}
\newcommand{\nscd}{\textsf{NSCD}}
%\Pisymbol{psy}{212} ou encore \Pisymbol{psy}{228}


\DeclareMathOperator{\rot}{rot}
\DeclareMathOperator{\sh}{sh}
\DeclareMathOperator{\ch}{ch}
%\DeclareMathOperator{\th}{th}
\DeclareMathOperator{\arcsh}{arcsh}
\DeclareMathOperator{\argth}{argth}
\DeclareMathOperator{\proj}{proj}


%%The Principal Value Integral symbol
\def\Xint#1{\mathchoice
   {\XXint\displaystyle\textstyle{#1}}%
   {\XXint\textstyle\scriptstyle{#1}}%
   {\XXint\scriptstyle\scriptscriptstyle{#1}}%
   {\XXint\scriptscriptstyle\scriptscriptstyle{#1}}%
   \!\int}
\def\XXint#1#2#3{{\setbox0=\hbox{$#1{#2#3}{\int}$}
     \vcenter{\hbox{$#2#3$}}\kern-.5\wd0}}
\def\ddashint{\Xint=}
\def\dashint{\Xint-}


%----------------------------------------------------------------------
%             Des chiffres avec des ronds autour
%----------------------------------------------------------------------
\def\nombrecercle#1{\def\taille{0.3}
                \put(0,0){#1}
                \put(0.08,0.08){\circle{\taille}}}



\def\ae#1{\stackrel{\mbox{\scriptsize a.e.}}{#1}}
\def\argmin{\mathop{\rm argmin}}
\def\eqref#1{{\rm (\ref{#1})\/}}
\def\indicfon{\mathord{\rm i}}       %indicator function
\def\p{\mathord{\rm proj}}
\def\N{\mathord{\rm N}}
% \def\prosca#1#2{#1\cdot#2}
\def\prosca#1#2{\langle #1,#2\rangle}
\def\qedtext{\mbox{}\hfill$\Box$}
\def\qedmath{\eqno\Box}
\def\s{{$\mathcal{S}$}}
\def\somme{\mathop{\textstyle\sum}}
\def\somme{\mathop{\textstyle\sum}}
\def\submoins{_{\scriptscriptstyle-}}
\def\subplus{_{\scriptscriptstyle+}}
\def\T{\mathord{\rm T}}
\newcommand{\RR}{\ensuremath{\rm\sf I\!R}}
\newcommand{\NN}{\ensuremath{\rm\sf I\!N}}



%----------------------------------------------------------------------
%             Macro M Jean 
%----------------------------------------------------------------------

\def\Real{\mbox{I\hspace{-.15em}R}}
\def\Integer{\mbox{I\hspace{-.15em}N}}
\def\Bunit{\mbox{I\hspace{-.15em}B}}
\def\real{\mbox{\scriptsize I\hspace{-.15em}R}}
\def\bunit{\mbox{\scriptsize I\hspace{-.15em}B}}
\def\IL{\mbox{\scriptsize I\hspace{-.15em}L}}
\def\Indic{\mbox{\large $\psi$}}
\def\bfxi{\mbox{$\xi$ \hspace{-1.1em} $\xi$}}
%\def\bfXi{\mbox{$\Xi$ \hspace{-1.1em} $\Xi$}}
\def\RunR{\mathcal R}
\def\RunRN{\mathcal R_{N}}
\def\RunRT{\mathcal R_{T}}
\def\RunS{\mathcal S}
\def\RunSN{\mathcal S_{N}}
\def\RunST{\mathcal S_{T}}
\def\RunU{\mathcal U}
\def\RunUN{\mathcal U_{N}}
\def\RunUT{\mathcal U_{T}}
\def\RunUP{\mathcal U'}
\def\RunUPN{\mathcal U'_{N}}
\def\RunUPT{\mathcal U'_{T}}
\def\RunJ{\mathcal J}
\def\RunW{\mathcal W}
\def\RunF{f}
\def\RunFa{f_{1}}
\def\RunFb{f_{2}}
\def\RunFP{f'}
\def\RunV{v}
\def\RunVP{v'}
\def\EspF{\mathcal F}
\def\EspV{\mathcal V}
%%%%
\catcode`\�=13
\def�{\'e}
\catcode`\�=13
\def�{\`e}
\catcode`\�=13
\def�{\`a}
\catcode`\�=13
\def�{\c c}
\def\N{\mbox{I\hspace{ -.15em}N}}
\def\Z{\mbox{Z\hspace{ -.3em}Z}}
\def\Q{\mbox{l\hspace{ -.47em}Q}}
\def\R{\mbox{l\hspace{ -.15em}R}}
\def\F{\mbox{l\hspace{ -.15em}F}}
\def\E{\mbox{l\hspace{ -.15em}E}}
\def\LMGC90{{\small \it LMGC90 }}
\def\NSCD{{\small \it NSCD }}
\def\CHIC{{\small \it CHIC }}
\def\half{{\frac{_{1}}{^{2}}}}
\def\12T{{\frac{_{1}}{^{2T}}}}

\def\geq{\geqslant}
\def\leq{\leqslant}

\begingroup
\count0=\time \divide\count0by60 % Hour
\count2=\count0 \multiply\count2by-60 \advance\count2by\time
% Min
\def\2#1{\ifnum#1<10 0\fi\the#1}
\xdef\isodayandtime{\the\year-\2\month-\2\day\space\2{\count0}:%
\2{\count2}}
\endgroup



\title{3DFClib: a collection  of discrete 3D Frictional Contact (3DFC) problems}
\author{3DFClib team}
\date{\today}

\begin{document}

\maketitle

\tableofcontents


\section*{Purpose of the document}



The goal of this work is to set up a collection of 3D Frictional Contact (3DFC) problems. The collection will provide a standard framework for testing available and new algorithms for solving discrete frictional contact problems. 

\section*{Notation}
Let us denote by the integer $n_c$ the number of contacts. The integer $n$ is the number of degree of freedom of the system and $m = 3 n_c$ the number of unknown variables at contacts.

For each contact $\alpha \in \{1,\ldots n_c\}$, the  local velocity  is denoted by $u^\alpha \in \RR^3$. Its normal part  is denoted by $u_\n^{\alpha}\in \RR$ and its tangential part $u_\t\in\RR^2$. One gets
\begin{equation}
  \label{eq:contactvelocity}
  u^\alpha =\left[
  \begin{array}{c}
    u^\alpha_{\n} \\
    u^\alpha_{\t}   
  \end{array}\right]
\end{equation}

The vector $u$ collects all the local velocity at each contact
\begin{equation}
  \label{eq:normal}
  u = [[u^\alpha]^T, \alpha = 1\ldots n_c]^T
\end{equation}
respectively for the normal part $u_\n$
\begin{equation}
  \label{eq:tangent}
  u_\n = [ u^\alpha_{\n}, \alpha = 1\ldots n_c]^T,
\end{equation}
and its tangential a part as 
\begin{equation}
  \label{eq:tangent}
  u_\t = [ [u^\alpha_{\t}]^T, \alpha = 1\ldots n_c]^T.
\end{equation}

for a contact $\alpha $, the modified local velocity, denoted by $\hat u^\alpha $, is defined by
\begin{equation}
  \label{eq:modified}
  \hat u^\alpha = u^\alpha +\left[
  \begin{array}{c}
    \mu \|u^\alpha_\t\| \\
    0 \\
    0
  \end{array}\right]^T
\end{equation}
The vector $\hat u$ collects all the modified local velocity at each contact
\begin{equation}
  \label{eq:normal}
  \hat u = [[\hat u^\alpha]^T, \alpha = 1\ldots n_c]^T
\end{equation}

For each contact $\alpha$, the reaction vector $r^\alpha\in \RR^3$ is also decomposed in its normal part $r_\n^{\alpha}\in \RR$ and the tangential part $r_\t\in\RR^2$ as
\begin{equation}
  \label{eq:contactvelocity}
  r^\alpha = \left[
  \begin{array}{c}
    r^\alpha_{\n} \\
    r^\alpha_{\t}   
  \end{array}\right]
\end{equation}
The Coulomb friction cone for a  contact $\alpha$ is defined by 
\begin{equation}
  \label{eq:CCC}
  C_{\mu^\alpha}^{\alpha}  = \{r^\alpha, \|r^\alpha_\t \| \leq \mu^\alpha |r^\alpha_\n| \}
\end{equation}
and the set $C^{\alpha,\star}_{\mu^\alpha}$ is its dual.


The set $C_{\mu}$ is the cartesian product of Coulomb's friction cone at each contact, that 
\begin{equation}
  \label{eq:CC}
  C_{\mu} = \prod_{\alpha=1\ldots n_c} C_{\mu^\alpha}^{\alpha} 
\end{equation}
and $C^\star_{\mu}$ is dual.
\clearpage
\section{Linear discrete problems with Coulomb's friction and unilateral contact}


\newtheorem{definition}{Definition}

\subsection{Reduced discrete problem. 3DFC problem}
\begin{definition}[Frictional contact problem (3DFC)]\index{mFC3D}
  Given
  \begin{itemize}
    \item a symmetric positive semi--definite  matrix ${W} \in \nbR^{m \times m}$
    \item a vector $ {q} \in \nbR^m$,
    \item a vector of coefficients of friction $\mu \in \RR^{n_c}$
  \end{itemize}
 the  3DFC problem  is to find two vectors $u\in\RR^m$ and $r\in \RR^m$, denoted by $\mathrm{3DFC}(W,q,\mu)$  such that
\begin{equation}\label{eq:lcp1}
  \begin{cases}
    \hat u = W r + q +\left[
      \left[\begin{array}{c}
          \mu^\alpha \|u^\alpha_\t\|\\
          0 \\
          0
        \end{array}\right]^T, \alpha = 1 \ldots n_c
    \right]^T \\ \\
    C^\star_{\mu} \ni {\hat u} \perp r \in C_{\mu}
  \end{cases}
\end{equation}
\qed
\end{definition}

\subsection{Global/local discrete problem. G3DFC problem}
\begin{definition}[Global 3DFrictional contact  problem (G3DFC)]\index{G3DFC}
  Given
  \begin{itemize}
    \item a symmetric positive definite matrix ${M} \in \nbR^{n \times n}$
    \item a vector $ {f} \in \nbR^n$,
    \item a matrix  ${H} \in \nbR^{n \times m}$
    \item a vector $w \in \RR^{m}$,
    \item a vector of coefficients of friction $\mu \in \RR^{n_c}$
  \end{itemize}
 the Global 3DFC problem  is to find three vectors $ {v} \in \nbR^n$, $u\in\RR^m$ and $r\in \RR^m$, denoted by $\mathrm{G3DFC}(M,H,f,w,\mu)$  such that
\begin{equation}\label{eq:lcp1}
  \begin{cases}
    M v = {H} {r} + {f} \\ \\
    \hat u = H^T v + w +\left[
      \left[\begin{array}{c}
        \mu^\alpha \|u^\alpha_\t\|\\
        0 \\
        0
      \end{array}\right]^T, \alpha = 1 \ldots n_c
\right]^T \\ \\
    C^\star_{\mu} \ni {\hat u} \perp r \in C_{\mu}
  \end{cases}
\end{equation}
\qed
\end{definition}




\subsection{Global  Mixed Frictional contact problem (GM3DFC)}
\begin{definition}[Global  Mixed Frictional contact problem (GM3DFC)]\index{GN3DFC}
  Given
  \begin{itemize}
    \item a symmetric positive definite matrix ${M} \in \nbR^{n \times n}$
    \item a vector $ {f} \in \nbR^n$,
    \item a matrix  ${H} \in \nbR^{n \times m}$
    \item a matrix  ${G} \in \nbR^{n \times p}$
     \item a vector $w \in \RR^{m}$,
     \item a vector $b \in \RR^{p}$,
    \item a vector of coefficients of friction $\mu \in \RR^{n_c}$
  \end{itemize}
 the Global Mixed 3DFC problem  is to find four vectors $ {v} \in \nbR^n$, $u\in\RR^m$, $r\in \RR^m$ and $\lambda \in \RR^p$ denoted by $\mathrm{GM3DFC}(M,H,G,w,b,\mu)$  such that
\begin{equation}\label{eq:lcp1}
  \begin{cases}
    M v = {H} {r} + G\lambda + {f} \\ \\
    G^T v +b =0 \\ \\
    \hat u = H^T v + w +\left[
      \left[\begin{array}{c}
        \mu \|u^\alpha_\t\|\\
        0 \\
        0
      \end{array}\right]^T, \alpha = 1 \ldots n_c
\right]^T \\ \\
    C^\star_{\mu} \ni {\hat u} \perp r \in C_{\mu}
  \end{cases}
\end{equation}
\qed
\end{definition}
\subsection{ Mixed Frictional contact problem (M3DFC)}
\begin{definition}[Mixed 3DFrictional contact  problem (M3DFC)]\index{M3DFC}
  Given
  \begin{itemize}
    \item a positive semi--definite matrix  ${W} \in \nbR^{m \times m}$
    \item a matrix  ${V} \in \nbR^{m \times p}$
    \item a matrix  ${R} \in \nbR^{p \times p}$
     \item a vector $q \in \RR^{m}$,
     \item a vector $s \in \RR^{p}$,
    \item a vector of coefficients of friction $\mu \in \RR^{n_c}$
  \end{itemize}
 the  Mixed 3DFC problem  is to find three vectors $u\in\RR^m$, $r\in \RR^m$ and $\lambda \in \RR^p$ denoted by $\mathrm{M3DFC}(R,V,W,q,s,\mu)$  such that
\begin{equation}\label{eq:lcp1}
  \begin{cases}
    V^T {r} + R \lambda + s = 0 \\ \\
    \hat u = W {r}    + V\lambda  + q +\left[
      \left[\begin{array}{c}
        \mu^\alpha \|u^\alpha_\t\|\\
        0 \\
        0
      \end{array}\right]^T, \alpha = 1 \ldots n_c
\right]^T \\ \\
    C^\star_{\mu} \ni {\hat u} \perp r \in C_{\mu}
  \end{cases}
\end{equation}
\qed
\end{definition}



\subsection{Remarks}


Note that the previous problems may be an instance of quasi-static problems: the matrix $M$ plays the role of the stiffness matrix and the vector $u$ is a position or a displacement. 
\section{Measuring errors}

\section{Detailed implementation}

\subsection{File format}

The proposed file format for storing and managing data is the HDF5 data format\\
 \url{http://www.hdfgroup.org/HDF5}


\noindent The data name should be defined as close as possible to the definition of this document.

\subsection{Matrix storage}
Three matrix storages are considered :
\begin{enumerate}
\item dense format
\item sparse format : row compressed format
\item sparse matrix of 3x3 dense matrices. (sparse block matrix)
\end{enumerate}

The last format could be deduced from the standard sparse format. To be discussed. 


\subsection{C implementation}

A C implementation will be proposed for reading and writing each of 3DFC problems into HDF5 files.

The storage of dense matrices will be in column major mode (FORTRAN mode).
\clearpage

\section{Additional  description of the problems}

The following additional information should be added in a reference document and in the HDF5 file.


\begin{itemize}
\item \verb?TITLE? : a title for the problem
\item \verb?DESCRIPTION? : The field of application. Short description on how the problem is generated.
\item \verb?MATRIX_INFO? : The sparsity and the conditioning of the matrices.
\item \verb?MATH_INFO? : Existence, uniqueness of solutions.
\item \ldots
\end{itemize}

\noindent The following data can be optionally added in the HDF5 file
\begin{itemize}
\item \verb?SOLUTION? : A reference solution
\item \verb?INITIAL_GUESS? : A initial guess
\item \ldots
\end{itemize}



\section{List of problems}

\end{document}



%%% Local Variables: 
%%% mode: latex
%%% TeX-master: t
%%% End: 
